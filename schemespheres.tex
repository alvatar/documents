% THIS IS SIGPROC-SP.TEX - VERSION 3.1
% WORKS WITH V3.2SP OF ACM_PROC_ARTICLE-SP.CLS
% APRIL 2009
%
% It is an example file showing how to use the 'acm_proc_article-sp.cls' V3.2SP
% LaTeX2e document class file for Conference Proceedings submissions.
% ----------------------------------------------------------------------------------------------------------------
% This .tex file (and associated .cls V3.2SP) *DOES NOT* produce:
%       1) The Permission Statement
%       2) The Conference (location) Info information
%       3) The Copyright Line with ACM data
%       4) Page numbering
% ---------------------------------------------------------------------------------------------------------------
% It is an example which *does* use the .bib file (from which the .bbl file
% is produced).
% REMEMBER HOWEVER: After having produced the .bbl file,
% and prior to final submission,
% you need to 'insert'  your .bbl file into your source .tex file so as to provide
% ONE 'self-contained' source file.
%

\documentclass{acm_proc_article-sp}

\begin{document}

\title{Creating multiplatform and mobile apps in Scheme with Gambit/SchemeSpheres}
%\titlenote{(Does NOT produce the permission block, copyright information nor page numbering). For use with ACM\_PROC\_ARTICLE-SP.CLS. Supported by ACM.}}
\subtitle{[Tutorial abstract]}
%\titlenote{A full version of this paper is available as
%
% You need the command \numberofauthors to handle the 'placement
% and alignment' of the authors beneath the title.
%
% For aesthetic reasons, we recommend 'three authors at a time'
% i.e. three 'name/affiliation blocks' be placed beneath the title.
%
% NOTE: You are NOT restricted in how many 'rows' of
% "name/affiliations" may appear. We just ask that you restrict
% the number of 'columns' to three.
%
% Because of the available 'opening page real-estate'
% we ask you to refrain from putting more than six authors
% (two rows with three columns) beneath the article title.
% More than six makes the first-page appear very cluttered indeed.
%
% Use the \alignauthor commands to handle the names
% and affiliations for an 'aesthetic maximum' of six authors.
% Add names, affiliations, addresses for
% the seventh etc. author(s) as the argument for the
% \additionalauthors command.
% These 'additional authors' will be output/set for you
% without further effort on your part as the last section in
% the body of your article BEFORE References or any Appendices.

\numberofauthors{1} %  in this sample file, there are a *total*
% of EIGHT authors. SIX appear on the 'first-page' (for formatting
% reasons) and the remaining two appear in the \additionalauthors section.
%
\author{
% You can go ahead and credit any number of authors here,
% e.g. one 'row of three' or two rows (consisting of one row of three
% and a second row of one, two or three).
%
% The command \alignauthor (no curly braces needed) should
% precede each author name, affiliation/snail-mail address and
% e-mail address. Additionally, tag each line of
% affiliation/address with \affaddr, and tag the
% e-mail address with \email.
%
% 1st. author
\alignauthor
Álvaro Castro-Castilla\\
       \affaddr{Fourthbit}\\
       \affaddr{Madrid, Spain}\\
       \email{alvaro.castro.castilla@fourthbit.com}
}
\date{22 May 2014}

\maketitle
\begin{abstract}

Mlkshk 90's sustainable kale chips tousled. Swag mumblecore blog Banksy, try-hard wayfarers Helvetica tousled art party fashion axe plaid Truffaut Kickstarter Neutra fixie. Semiotics bespoke tofu butcher. Yr cliche crucifix, put a bird on it Neutra Portland authentic kale chips leggings Marfa dreamcatcher. Narwhal bicycle rights Bushwick cred. YOLO literally put a bird on it pour-over. Polaroid whatever paleo food truck.

%Target: build native multiplatform apps in Scheme with a set of tools that handle:

%* Project construction, in all development platforms, for the different targets available in each.
%* the different project structures and build processes for all platforms, without hiding the details.
%* dependency and some form of modules / packetization, so the system favours reusability and code sharing.
%* providing a set of libraries that supports most common needs for all target platforms.

\end{abstract}

% A category with the (minimum) three required fields
\category{D.2.7}{Software Engineering}{Distribution, Maintenance, and Enhancement}
%A category including the fourth, optional field follows...
\category{D.2.m}{Software Engineering}{Miscellaneous}[rapid prototyping, reusable software]

\terms{Design, Documentation, Experimentation, Languages}

\keywords{Scheme Programming Language, multiplatform, mobile, Gambit, SchemeSpheres, automation, software construction, framework} % NOT required for Proceedings


\section{Introduction}

\subsection{Motivation}

%Move1: Outlining / promoting / problematizing
Implementation fragmentation and diversity has been a long-standing part of the Scheme community. \textit{Scheme Now!}, \textit{Eggs}, \textit{PLaneT} are the most prominent proposals aiming at defining a package specification for shared code, as well as centralizing packages in common repository. Specially the case of \textit{Scheme Now!}, which has been developed with the intention of serving as a cross-implementation system, might be seen as an approach to the fragmentation problem. Most other package systems aim at solving the distribution and sharing problem for a specific implementation and Scheme ecosystem.

%Move2: Justifying this particular piece of study/research
With the advent of mobile technologies, complexity grows as the problem becomes not only of distribution, but of software construction workflows. Building, debugging and deploying code for mobile platforms vary wildly in aspects such as the platform's SDK API language and project structure. Moreover, mobile platforms usually involve very particular toolchains that are rapidly evolving over time. With these new added complexities, it becomes particularly hard to make an all-in-one solution that becomes sufficiently accepted by the community. More so if it's supposed to work for a wide range of Scheme systems seamlessly.

Taking these issues into account, the main motivation behind SchemeSpheres is providing a framework for rapidly building applications for multiple platforms. This functionality is similar to what package systems and central repositories provide, but more th emphasis is placed on the workflow and construction, instead of distribution. SchemeSpheres is focused on the project setup and construction, simplifying the creation of projects for multiple platforms and providing a set of tools for the build/debug/release cycle. It is sufficiently flexible to allow the Scheme developer to choose the right libraries and design the application architecture freely, thus preserving handling complete control to the final suer. For this to become possible, a working framework should provide the following key features:

\begin{itemize}
  \item Project construction automation, in all development platforms, for the different targets available in each.
  \item Project structure templates and build processes for all platforms, without hiding the details and allowing full customization per-project.
  \item Dependency handling and some form of package specification, so the system favours reusability and code sharing.
  \item A common documentation system, manuals and guides.
  \item Additionally, a set of libraries that supports most common needs for all target platforms.
\end{itemize}


Mainly:

* Creating a multiplatform project involves many more aspects than just generating the code. Making it easy and readily available is very beneficial to the community.
* Structuring, documentation and reusability also help in the process of building software.

Other things to take into consideration:

* The use of many SRFIs implies syntax-rules. Syntax-rules must be handled by the building system as well.
* Other libraries require define-macro or syntax-case.
* Short names can collide easily.





\section{Proposed framework}

%Move 3: Methodological, demographic, or procedural comments. Approach.

\subsection{Requirements}
Needs:
* Support for syntax-rules
* Support for low-level macros (define-macros/syntax-case/rsc-macro-transformer), but those macros should be used only without interacting with syntax-rules.
* Modules
* Complete control over the workflow and build process
* An R5RS+ Scheme compiler that can generate native or portable C code for the target platforms. Gambit has been chosen for this task.

\subsection{Approach}

1. Syntax expansion, modules, conditional compilation:

* syntax-rules mainly
* low-level for isolated cases where it is really necessary to support a library
* modules missing
* cond-expand for conditional compilation


2. SchemeSpheres is based on the workflow comprised of three tools.

* Sfusion scaffolds out a new application, writing the Ssake configuration and pulling in relevant Grunt tasks and Bower dependencies that you might need for your build.

* Ssake is used to build, preview and test your project, thanks to help from tasks curated by the Yeoman team and grunt-contrib. A program to handle tasks (task runner), with procedures for build automatization in different platforms.

* Sspheres is used for dependency management, so that you no longer have to manually download and manage your scripts.


3. (proto)Ecosystem:

* An structuration and curation of modules into Spheres, with sharing of common and uniform functionality (such as FFI, ...)
* documentation of the process
* A central repository




\section{Example of application building}




\section{Findings and gotchas}

%Move 4: Summarizing the main findings. HOW TO DO:


How to do:
* the generators approach? did it work?
* handling conditional compilation (cond-expand)


Main issues:
* syntax expansion and proper error reporting.



%Move 5: Highlighting its outcome/results

* How easy it is
* What advantages it has


\section{Conclusions and Roadmap}

%Move 6: Further observations (implications, limitations, future developments)

* Roadmap:
  * Modules
  * Javascript backend
* Limitations


%APPENDICES are optional
%\balancecolumns
\appendix
%Appendix A
\section{Headings in Appendices}
The rules about hierarchical headings discussed above for
\balancecolumns
% That's all folks!
\end{document}
